\documentclass{article}
\usepackage{framed}
\author{Steve Portigal}
\title{O'Reilly Webcast 11/21/2013}

\begin{document}
\maketitle

{\bf discover and act} on new customers and themselves

"ethnography": loaded word, not always describe what communicate

generic: not committing to methodology
there are best practices but always iterating and improving
discover and act

THEMSELVES: what's unspoken, unknown in own culutre
not vaccuum
something new or different built on starting point
anchor activity in who we are as a team

distracting:
whatever!

\begin{framed}
\begin{itemize}
\item{ethnography}
\item{ethnographic interviews}
\item{video enthnography}
\item{depth-interviews}
\item{contextual research}
\item{home visits}
\item{site visits}
\item{experience modeling}
\item{design research}
\item{user research}
\item{user-centered design}
\item{one-on-ones}
\item{camer studies}
\item{user safaris}
\end{itemize}
\end{framed}

unpacking the language

---

\section{What are we doing?}
\subsection{Examine people in their own context}

their kitchen, shop, factory, car: where they are is key
some kind of examination: generic term, with specific details of approach customized

what are they doing?
many often stuck?
analytical

process, expectations, behavior

what does it mean?
how do people construct values, stories, relationships
hidden
meaning from doing but important to remember meaning is crucial!
dive in!


\subsection{Infer}
interpret synthesize
find the connections
the researcher is the apparatus

not requirement gathering, much messier, creative, not just scooping
not really making sense on the surface

\subsection{Apply to problem}
business, service, etc
tell a story by going further than started
create a X or fix a Q but level of insight points to other kinds of solutions
real problem: IS HERE, not just X or Q
additional opportunities
deeper need

\section{Methods}
{\it Take a fresh look at people}
\subsection{what to make or do}

{\it Use existing ideas as hypotheses}
\subsection{Refine and prototype}
varying fidelity - can interview at any point
through lense of solution
talk about both problem and solution, provoking, testing hypotheses, exploring

don't test solutions, test hypotheses


\subsection{Launch}

\subsection{Iterate and improve}
{\it Explore new ideas}



\section{Questions}

what kinds of questions do we want to ask?

Business: What does the business what to do?
Research: What do we need to understand in order to support that?
Methods: What are the methods to use to support research to support business

surface business question
needs to be facilitated
both evolve, even in spite of timelines
not really hierarchical

tightly coupled pieces

\section{Pain Points}
"default research/business question"

While we always uncover so-called pain points, the bigger oppportunity may come from understanding why - how did we get here?

Drawers overflowing with obsolete electronic products - if this is a pain point, then we make bigger drawers for their storage of discarded tech
Limited lense

Real problem: why is tech obsolete? how do we buy it? how does it trickle down? consumption? current?

not just bigger drawers

--

paint can: not fully realized but tolerant
pain point is drips
can is not self opening, paint key needed, aftermarket lid bad
complicated systemic problem in supply, user, expectations

doesn't solve the larger problem
be cautious about using pain points as your lense

\subsection{Satisficing}
Herber Simon 1956

"Acceptance of good-enough solutions: satisify and suffice"

Good but not great solutions

Design: incredible delight: but people are tolerant of good enough
effort to make stupendous exceeds the pain, it's not GOING TO HAPPEN

designers and engineers go crazy, want to fix, make a certain way
but world doesn't care? real problem isn't always what it appears to be

everything about printing: what came up was her child, her husband's affair, this was good enough, was never going to replace that cable

for us to go off to think we're solving these problems without understanding where people are coming from is a risk...

be more immersed in empathy

when is that a good solution for user to solve the problem? understand their own perspective, not just your pain point!

we do more than find pain points and collect them

\section{methodology}

survey with large sample size

talked with different music programs, most of them could not play a music file
all very controlled

in-home interviews

not this vs that

\section{mechanics}
Who do you include?
Who you learn from doesn't need to be the customer

by creating contrast, you reveal key influencing factors that you otherwise wouldn't see

triangulate through multiple perspectives

\begin{itemize}
\item{typical user}
\item{non-user}
\item{extreme user}
\item{peripheral user}
\item{expert user}
\item{subject-matter expert}
\item{wannabe user}
\item{should-be user}
\item{future user}
\item{past user}
\item{hater}
\item{loyal to competitor}
\item{teacher}
\end{itemize}

\subsection{type of user}
Think about the whole system: chooser, influencer, user, impacted

Challenge assumptions about hwo the organization is implicitly/explicitly designing for
is that everyone?
do they even exist?

This will surface a broader sense, ever prior to research, about who is affected by the product and who it's being designed for

\section{field/interview guide}
A detailed plan of what will happen in the inteview
questions, timing activiies tasks logistics

\begin{framed}
"questions we want answers to"
"questions we will ask"
\end{framed}

have a conversation, open ended, move people through things they haven't thought before, set them up to answer those questions without asking

\paragraph{share with team to align on issues of concern}
stakeholders everyone
especially with multiple teams in the field

helps you visualize the flow of the session
include questions as well as other methods that you'll use


\section{4 questions}
\subsection{introduction and participant bg}
logistics, timing objectives
\subsection{the main body}
overview
\subsection{projection/dream questions}
richer, be audacious and ask about predictions for future or idea
\subsection{wrap up}
any questions

history and bg
shopping and learning
integration into physical environment
connectivity
wishlist and future features

show relative weight by adding time
how many probes: why and why not
not algo, set you up

\begin{itemize}
\item{tasks}
\item{participation}
\item{demonstration}
\item{role-play}
\item{observations}
\end{itemize}

\subsection{Participatory design}
give people the chance to solve problem, doesn't mean we implement the requested solution literally
generate alternative
solve underlying need

\subsection{Show solution}
testing vs exploring
build provocations: surface hidden desires and expectations
what the solution could be, what the problem space is in way they couldn't

\subsection{Make sure you are asking the right questions}
what does this solution enable? what problems does it solve?

\begin{itemize}
\item{mockups}
\item{working prototypes}
\end{itemize}

\section{Range of methods}
logging
homework
stimuli
exercises

start creating library, make or mashup new pieces

casual card sort
online reviews - a visual thing to move around, a warmup to do your own thang

\section{Observe the culture}
elements of what the rules are in a society
notice how cultural artifacts reflect and define environment, what is normal

norms define what people choose or ignore
\begin{itemize}
\item{media}
\item{products}
\item{ads}
\item{street culture}
\item{trends/fads}
\end{itemize}

things that symbolize norms: barriers, times where people will or won't do something you want - go into context takes us through culture that may be unfamiliary

\section{documentation}
\subsection{photos}
take a lot of photos
they will reveal things you don't remember noticing
don't snap until everyone comfy
essential for storytelling
need permission

\subsection{audio video notes}
essential for exactly what is said
difficult to maintain eye contact, manage interview, write down everything
notes can help

\section{BEST PRACTICES}

\subsection{check worldview at door}
braindump all the assumptions and expectations, get it out of your head, ndon't verify assumptions. make interview about the interview. LEARN ABOUT PAUL
\subsection{embrace how others see the world}
go where your users are rather than asking them to come to you. nip distractions. eat! leave plenty of time! find bathroom before!

ask questions that you think you know the answer to
what do you know, what are you afraid they'll say, what might you learn?

ask and embrace


\subsection{build rapport}
be selective about social graces, just enough small talk, accept what you're offered

be selective about talking about yourself
reveal personal information to give them permission to share
otherwise think OMG ME TOO without saying it
be enthusiastic and engage without telling about you, but can use it to get unstuck
wasn't relevant until... normalizing! reassuring!

work towards the tipping point
from question-answer to question-story
you won't know when it's coming, be patient - eventually you'll get the stories

\subsection{listen}
you can demonstrate listening by asking questions
follow-up
use earlier
i want to go back to x

signal your transitions
"great now i'd like to move on"

level of listening is not how we normally talk to each other
you are interviewing, not having a conversation, it's hard

\subsection{charismaaaaaa}
listening body language

\subsection{silence defeats awkwardness}
after you ask your question, be silent. don't put the answers in the question!
answer they answered you, be silent
{\bf the more times you tell them implicitly what you want the answer to be, the harder it is for the other}
don't help them answer, give them space
don't hold onto your talking
"let there be silence" just wait just nod let them keep going

\subsection{use natural language}
talk like your subject talks
even if they're wrong, don't be implicitly damning by correcting
this about them being the expert

\subsection{fix something? wait until the end}
to eff it up is to by fixing it yourself
it's frustrating to watch users struggle but you are there to remember from them

you will lose the interview if you start taking their questions

when it's time to go, show or tell what will help THEM

\subsection{prepare for exploding questions}
coping techniques
wait until these issues come up organically
make notes on your field back about what you want to loop back
triage based on what's most pressing
triage based on what makes the best follow-up to demonstrate listening: EG EMOTIONAL CUES
tesing: something that's really important to me, do you get where i'm coming from?

exploding questions can lead to a flow state
interview jumping around, both fast and slow
enjoy that

\subsection{we learn from mistakes and mishaps}
collect and share war stories from other interviewers
real, unusual messiness from this process

\section{extra}
present yourself as legitimate

how do you get the right users to interview?
contrast
figure out the problem
brainstorming who will give insight on that
who can we get, practically?
tactical: behavior rather than attitude
"we want to find people who are hip" : NOPE
"we miiight find it in a person who shops at sephora"
objective criteria
tweets more than 40 times a day vs a twitter account occasional
meeeeat



\end{document}