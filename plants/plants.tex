\documentclass{book}

\begin{document}
Pot your sansevieria plant correctly.
Use a good houseplant potting medium, not garden soil.
Repot only when the plant starts breaking the pot with its roots.

 
2Place your sansevieria plant in the right light.
Put your sansevieria plant in an east, west or north windowsill any time of the year.
Put the sansevieria plant in a south window during the winter months or 12” (30.5 cm) away from a south window in late spring and summer.
Provide bright florescent or other lighting and sansevieria plants will do fine.
3Water the sansevieria plant correctly.
Use room temperature water.
Use distilled or rain water if possible.
Let the pot surface feel dry to the touch before watering in spring and summer.
Water very little in winter or in a cool air-conditioned room. Wait until the pot is quite dry before watering.
Water if you notice the leaves are drooping and the pot feels dry.
Water along the sides of the plant and try to keep water out of the center of the leaf clump.
Water until water drains from the bottom and empty drained water from trays promptly.
4Fertilize sansevieria plants once in the spring with houseplant fertilizer mixed according to label directions.
5Keep plants in temperatures between 40º and 85ºF (4.4 to 29.4 C).
6Wipe the leaves of sansevieria with a damp cloth if they get dusty.

http://www.wikihow.com/Care-for-a-Sansevieria-or-Snake-PlantSnake Plant 101 - Taking 'Easy' to a New Level

Snake plant (Sansevieria), A.K.A Mother-In-Law’s Tongue, is a succulent plant, which means the leaves retain lots of water, similar to a cactus. You can recognize snake plants by their long, pointed, upright leaves which look very much like snakes, giving the plant its name. It is green in color, but sometimes has yellow edges.

1) When it’s time to buy a new plant, be sure to select a healthy looking plant that is a deep green. Pale leaves mean the plant may already be suffering at the store, and there’s no reason to waste your money on snake plant that’s growing poorly before you even get it home!

2) Often times when you buy a new houseplant, you should repot it into a new pot with fresh soil ASAP. If the roots appear above the soil, the plant definitely needs to be repotted. Also, if the pot from the store is simply a cheap plastic pot, I recommend you repot your new plant. It will be much happier that way!

3) Give snake plant sufficient light. The plant will thrive in a sunny window facing east, west or south. If you have only a north facing window, don’t despair! Snake plant handles a variety of light situations very well. If you can only provide minimal light, the plant will be fine. Later, when the opportunity presents itself, move snake plant to a sunnier window. The more sun it gets, the faster it will grow!

4) Water sparingly. A succulent holds lots of water inside its leaves, and if you give it too much water, the plant will rot. By sparingly, I mean water every two to three weeks (In the summer time, the plant will use more water, especially if it's in a very sunny location. In winter, the plant uses less water, so be sure to adjust accordingly). Use room temperature water so the roots don’t get shocked. Wet the top of the soil thoroughly, but be sure not to drown the plant! You are better off watering the plant too little than too much!


5) Feed the plant with a tiny amount of general all purpose plant food (the kind you dissolve in water) in the spring and summer. Because this is the primary growing time for snake plant, it will thank you for the extra growth boost the fertilizer provides. Put the fertilizer in your favorite watering jug, and just fertilize every time you water. In fall and winter, stop fertilizing – snake plant doesn’t need to be fed at this time of year.

6) Snake plant is extremely forgiving, and in fact will thrive more on neglect rather than too much attention. If you miss a watering, don’t worry! Water the next time you remember. Snake plant will survive in low light conditions, or even full sun! Therefore, snake plant is one of the most common and easy-to-care-for plants you can buy.

Important tip: As I already mentioned, do not overwater! The plant will die if it can’t absorb excess water in the pot. Overwatering is the number one cause of plant death!

If you be sure to give good light and small amounts of water to snake plant about every two to three weeks, it will prosper and bring you joy for many years!

http://houseplantscare.blogspot.com/2010/10/snake-plant-taking-easy-to-new-level.html

Common Name: Snake Plant
Scientific Name: Sansevieria species
Lighting: Moderate to Bright Light
Watering: Low to Moderate


Snake Plant
The Snake plant is the ultimate for those without a green thumb. This house plant is one of the hardest to kill. A Snake plant contains heavy, sword-like leaves which shoot up from the base of the dirt. They will grow in a clump like style. Smaller shoots will eventually grow as well. Another common name for this plant is Mother-In-Law's Tongue.


The Snake plant is said to prefer moderate to bright light, however my experience is you can put it anywhere and it will grow. I had mine in an upstairs room with the blinds shut for weeks and it still was beautiful and it is still growing over 4 years later.


This houseplant prefers to be watered once every 7 - 10 days thoroughly, allowing the soil to dry in between waterings. If the foliage begins to droop, you are probably over watering. On the other hand if the foliages begins to wrinkle or bend over you are not providing enough water. This houseplant prefers to be pot bound, so avoid repotting unless the current pot is too small to keep upright.


On a special note, this houseplant is toxic when eaten. It is one of the many poisonous houseplants, so please keep away from pets and children.


Like I said earlier, this plant is really hard to kill, it's one of the easiest houseplants to grow. View my easy to grow houseplants list for other hardy choices.


http://houseplants-care.blogspot.com/2006/05/snake-plant-care.html
-------


The florist cyclamen is derived from Cyclamen persicum, a Mediterranean plant. In nature it goes dormant during the summer months, comes into growth as cooler, damper weather starts, flowers in autumn, winter or spring, and goes dormant again as the summer becomes warm. Cyclamens grow from tubers that are round and rather flat. The tubers are the storage organs that keep the plants alive during their summer dormancy.
When choosing a cyclamen be sure to select one with only a few flowers open. The flower stems should stand straight up, and there should be lots of buds tucked underneath the foliage that will develop and bloom later.

cyclamen
Photo: Deb Brown

Getting the Most Out of Your Blooming Plant

After you have just received a cyclamen, it's important to keep it cool and to water it correctly, making sure not to under- or overwater. To prevent disease problems, it is also a good idea to maintain good air movement around the plant.
From late autumn to early spring, provide your plant with as much light as possible. Sunburn is usually not an issue this time of year, although you may want to avoid placing your cyclamen directly in front of a south-facing window.

Aim for temperatures between 40° and 50°F at night and day temperatures less than 68°F. A cyclamen won't be too happy in a house heated much above 70°F, with the dry atmosphere that goes with it. If you are unable to provide cool enough conditions, the plant will survive for a time, but eventually it will develop yellow foliage and its blooming time may be cut short. It will probably tolerate a less than ideal location for a day or two as long as you return it to a better place shortly afterwards. The plant will tolerate indoor conditions even better if you move it to a cool spot at night. Make sure to provide as much light as possible in its daytime location.

Watering incorrectly can cause many problems, especially when too much water has been applied. Always wait until the soil surface feels dry before you water, but don't wait until the plant becomes limp. Do not water the center of the plant or the tuber may rot. A cyclamen prefers to receive a good soaking, then dry out partially before receiving a good soaking again. Allow the plant to drain over a sink or empty the water collection tray beneath the container after a few minutes. This will help prevent the roots from remaining too wet, which can lead to rotting.

Fertilize your cyclamen with a water-soluble fertilizer recommended for use on indoor plants, mixed half strength. Apply it every 3 or 4 weeks, starting about a month after you receive the plant. Overfeeding is more likely to produce foliage than flowers.

Dead flowers or leaves should be removed by giving their stems a sharp tug. If a sharp tug doesn't remove them, wait another day and try again. You don't want to risk yanking out a chunk of the tuber along with other healthy leaves.

http://www.extension.umn.edu/yardandgarden/ygbriefs/h145cyclamen.html



Care

Potting Soil: Cyclamen persicum does best planted in a soil-based potting mix, with the top of the tuber just slightly above the soil line.
Water:

When leaves are present, the plant is actively growing. Water whenever the soil feels dry. Avoid getting water on the crown of the plant.
As the flowers begin to fade, gradually allow the plant to dry out for 2-3 months. It's going into a dormant stage (see below) and any excess water will cause the tuber to rot.
New growth will probably start to appear around September. At this point, resume watering and feeding. Bring it back indoors before the cold weather.
Humidity: High humidity, especially during winter, is crucial. Keep the cyclamen on a tray of water with a layer of pebbles or something else to form a shelf for the cyclamen pot to sit on. Do not let the cyclamen itself sit in the water.

Fertilizer: Feed with a low-nitrogen fertilizer every couple of weeks while in full leaf.

Light: Give cyclamen bright, indirect light in the winter. While your plant is dormant during the summer, keep it out of bright light.

Temperature: Cyclamen do not like heat, but they are not frost hardy. Do not expose to temperatures below 50 degrees F. Avoid drafts as well as hot, dry air.



http://gardening.about.com/od/houseplants/a/Cyclamen.htm

Cyclamen are actually a type of bulb or more specifically a Corm (a short, thickened vertical stem). Their native habitat is the Mediterranean and Southern Europe. In your home, Cyclamen like to be a little on the cooler side with temperatures around 61˚F and in direct light or bright indirect light.

While in bloom, keep the root ball moist and feed the plant every two weeks. Cyclamen should be kept moist by watering in a tray and allowing the roots to take up the water rather than watering from above the plant which can lead to rotting. Remove yellow leaves and spent flowers.

When cyclamen are done blooming they can be discarded or the corm can be saved. After the foliage dies back, the plant should be left to dry. The corm should then be dug and repotted in midsummer and placed in a warm place so it can establish roots before returning it to a cool 55 - 60˚ F. to encourage flowering.



http://www.gertens.com/learn/Annuals-Perennials/cyclamen.htm

------

Purple Passion Plants need bright light for the brightest coloring but they must be protected from hot afternoon sun.
They prefer a slightly acidic soil mixture, so you should use a soil mix consisting of 2 parts peat moss to 1 part potting mix and 1 part coarse sand. The soil should be kept evenly moist at all times during the growing season.
Feed every 2-3 weeks with half-strength house plant food when the plant is actively growing.
Pinching the tips regularly will help produce a fuller plant.

http://www.thegardenhelper.com/gynura.html


Purple passion vine is an easy house plant to care for as long as it is neither overfed nor kept in a dark place.
The clambering branches and toothed leaves of the purple passion vine are so completely covered in purple hair that the entire plant takes on a rich, velvety appearance. The orange shaving-brush flowers are unattractive and unpleasantly scented and should be suppressed.
Don’t overfeed this plant or give it insufficient light -- its attractive purple coloration will fade. It also ages rapidly, so don’t hesitate to prune it severely. New plants are readily started from cuttings.
Purple Passion Vine Quick Facts
Scientific Name: Gynura aurantiaca sarmentosa
Common Names: Purple Passion Vine, Velvet Plant
Light Requirement for Purple Passion Vine: Bright Light to Filtered Light
Water Requirement for Purple Passion Vine: Drench, Let Dry
Humidity for Purple Passion Vine: High
Temperature for Purple Passion Vine: House to Cool
Fertilizer for Purple Passion Vine: Balanced
Potting Mix for Purple Passion Vine: All-Purpose
Propagation of Purple Passion Vine: Stem Cuttings
Decorative Use for Purple Passion Vine: Hanging Basket, Table
Care Rating for Purple Passion Vine: Easy


http://home.howstuffworks.com/purple-passion-vine.htm


The purple velvet plant (Gynura aurantiaca) is a tropical perennial that grows outdoors in Sunset Climate Zones H1 and H2. Its green foliage is covered in bright purple hair, which gives the plant its velvet-like texture. This stunning plant does not tolerate cold so it is generally grown as a houseplant outside of Hawaii. Unfortunately, the daisy-like blooms produced by the purple velvet plant are pollinated by flies and have an unpleasant odor. Despite this unpleasant aspect, which can be controlled with pruning, the purple velvet plant creates an eye-catching addition to the interior of your home.
Sponsored Link
Sambucol Black Elderberry
The Original Elderberry Extract Trusted by millions worldwide
Sambucolusa.com
1
Plant the purple velvet in a planting pot filled with potting soil rich with organic material. The planting pot must have several holes in the bottom to allow good drainage.
2
Place the purple velvet plant where it will receive 6 to 8 hours of bright, indirect light. Keep the plant about 3 to 6 feet away from windows. The purple velvet plant grows best when daytime temperatures are between 60 and 65 degrees and nighttime temperatures are 55 to 60 degrees. Plant growth will slow down when temperatures drop below 60 degrees.
3
Water the plant when the soil becomes slightly dry. Test the soil by inserting your finger several inches into the dirt. If the soil feels moist, do not water. If the soil feels dry, evenly moisten the potting soil with water. Allow tap water to warm to room temperature for 1 hour before watering. This will allow the chlorine typically in tap water to dissipate. Avoid getting the leaves wet when watering as this will encourage rot.
4
Apply a 3,1,3 NPK fertilizer to the purple velvet plant regularly as stated on the fertilizer usage instructions, which is typically once every 2 to 4 weeks. This will encourage plant growth. Reduce feeding during winter months.
5
Pinch off flower buds as they emerge with your fingertips to keep the purple velvet plant from blooming. This will help control the size of the plant as well as prevent the foul-smelling blooms from stinking up the air. Discard the flower buds in the trash. Prune leggy vines with a pair of clean, sharp pruning shears.
6
Monitor the purple velvet plant for pests such as mealy bugs or spider mites. These pests will cause yellowing of the leaves and premature leaf dropping. Mealy bugs look like small pieces of cotton; spider mites look like red dots and cover leaves in webbing. Treat the pests with insecticidal soap or horticultural oil.


http://homeguides.sfgate.com/grow-purple-velvet-plants-22935.html

----


Placement:   Since east windows allow direct sun rays and south windows allow indirect, but all day, light in consider putting the plant in a room that’s north or west facing instead. This is not to say the plant should be right on a window sill. Keeping the Peace Lily six to eight feet away from a window helps keep light levels consistent as well.

Light: Prefers medium, indirect sunlight. Yellow leaves indicate the Peace Lily is  getting too much light. Brown spot and streaks indicate direct sun rays have reached the plant and scorched it. Peace Lilies do fine under florescent lights and some have been known to thrive in rooms with no windows at all!

Temperature: The Peace Lily makes a great house plant because it thrives in the indoor temperatures people enjoy. A temperature range of 65°F to 80°F keep the plant happy. As cold drafts will harm it, make sure to keep it away from non-insulated windows and screen doors. Beware. Because temperatures below 45°F will kill it, so keep it indoors much of the year.

Fertilizer: While not every grower fertilizes the Peace Lily, those who want the best blooms make sure they fertilize during the spring and summer growing season. Feed a general house plant fertilizer (20-20-20) at one-half or one-quarter recommended strength once per month spring through summer.

Water:  Many growers wait for the plant to droop slightly before watering, allowing the plant to tell them when it’s thirsty. In general, water at least once a week and spritz the leaves with soft or distilled water throughout the summer growing season. Water plant less in winter.

If you find your plant completely depleted with fronds flat over pot edge because you forgot to water for a while, don’t just throw it in the trash. Water and spritz right away. You may be surprised at how quickly the Peace Lily revives.

If your municipal water system is heavily chlorinated, fill a container with water and then allow it to stand overnight so the chlorine can percolate out before pouring into the Peace Lily. Peace lilies can be sensitive to chlorine.

Growth:  The standard Peace Lily can grow to 24-40 inches and deluxe plants can grow to 32-50 inches.

Blooming:  The white blooms of the Peace Lily generally appear in the spring as more of a modified leaf, a “bract,” than a multi-petaled flower. Very well cared for plants may bloom again in the fall as well. Blooms last for two months or more. After blooms fade, a period of non-blooming follows.

Re-potting the Peace Lily:  Re-pot the Peace Lily when roots are apparent or when it seems to be taking up all water provided every few days. See re-potting section below.



http://www.proplants.com/guide/peace-lily-care-guide


Peace lilies, also known as closet plants, are one of the most popular varieties of houseplants. They are relatively easy to maintain, and their pure white blossoms are beautiful to look at. Learning how to care for peace lilies properly will keep them healthy, and ensure you get the most out of your plants.

 
EditSteps

1Choose a spot in the room that offers indirect sunlight. The peace lily can be close to but not directly under a window or near sunlight but not directly in it.

 

2Water the peace lily adequately. The best care you can give this plant is a heavy watering. When the potted soil is dry, add enough water to make it damp, but not so much as to have standing water.

3Mist the leaves several times a week with a spray bottle. Peace lilies thrive in higher humidity levels, so the more water you can supply the blooms, the healthier it will be.

4Trim any unhealthy leaves of your plant regularly. Use scissors to remove any unhealthy or dead spots to keep the foliage a lush green.

5Leave the lilies to grow on their own. Outside of water, your plant will not need much maintenance. Fertilizers and nutrient tablets are unnecessary.

 
EditVideo
 


EditTips
If your lilies develop brown tips on their leaves, the plant is not thriving, and you will need to determine the cause. Too much sunlight can be a factor, so move it to a more shaded spot. Water, too much or too little, may also be the culprit. Assess what may apply to your plant, and adjust your watering schedule accordingly. Also, check the temperature of the room it's in. Peace lilies do not like cold air and can wilt or die quickly if it is too chilly.
While your peace lily calls for "heavy water," it is possible to give it too much, essentially drowning it. To find the right balance, let the potting soil dry out completely after the first watering, before placing more water at or near the plant's base again. Once it's dry to your touch, the soil is ready to be watered again.
Direct sunlight is harmful to a peace lily as it can kill its foliage. If the leaves start to turn yellow, the plant may be receiving too much sunlight. Move it to a darker area.
Watch the leaves of your plant for signs of what it may need. If the leaves begin to droop, or the bottom-most leaves are turning yellow and wilt, you need to provide water.
http://www.wikihow.com/Care-for-Peace-Lilies


Light and Shade: Peace lilies love shade and some indirect light. A spot 5-7 feet away from a south- or west-facing window will often provide the right mix. Yellowing leaves, brown spots or streaks may mean that your plant is getting too much light, so try moving the plant a little further from the window or experiment with a new, less sunny location. Peace lilies can sometimes even do well under a fluorescent light without any sunlight at all! If you move your plant into a shadier location and its leaves are still brown, it may need a bit of misting on the leaves.
Temperature: Peace lilies like a consistent temperature between about 65 and 80 degrees F. Protect your plant from drafts and cold or drastic changes in temperature.
Fertilize: Spring and summer, use an organic fertilizer to help your plant bloom. Keep in mind that peace lilies are sensitive to chemical fertilizers, so organic options are best.
Water: Peace lilies like to be watered a lot at once, but also need a chance to dry out afterward. The plant will droop a bit when thirsty, telling you when it needs a drink! If you pay attention to when it usually starts to sag,  you can plan to water one day before it generally happens. Watering about once a week and spritzing leaves with water throughout the summer will help keep your peace lily hydrated. If your plant seems to completely droop, don’t give up: water and spritz and give it a chance to revive. If your water is chlorine-heavy, let a container of water stand overnight before watering the plant.
Drainage: Peace lilies are susceptible to root rot, so it’s very important to make sure the plant has a chance to dry out between waterings and that the container it lives in drains well. If your peace lily starts to wilt, check the roots to make sure they are firm and light-colored rather than soggy.
Transplant: When your plant’s roots show or your peace lily seems to be drinking up all its water within a few days, re-pot into a larger container. Your plant may need to be gradually moved into larger and larger containers, but generally peace lilies won’t need to go into a pot larger than 10 inches.
Expect your peace lily to show off its familiar white blooms in the spring. The plant is mildly toxic to animals and humans, so keep away from small children and Fluffy, and wash your hands after handling your plant as it produces crystals that can irritate skin. Follow these peace lily care tips and, with some experimentation, your peace lily should bring beauty to your house for years to come!

- See more at: http://www.hgtvgardens.com/houseplants/give-peace-a-chance-peace-lily-care-tips

\end{document}