\documentclass{article}
\title{How to Stop Laughing After Every Comment}
\author{Edited by Glutted, Emma Elfeirr, Tom Viren, Flickety and 24 others}

\begin{document}

\maketitle

\section{Introduction}
A giggle here, a guffaw there and you can usually be forgiven for having a hearty sense of humor about life and the things that happen to you. However, if you can't restrain yourself from laughing after everything someone else says, it's likely that you're not so much imbued with a sense of humor as suffering from a sense of inferiority, discomfort around others or just poor reading of social cues. Laughing every time someone comments soon comes off as annoying, offensive or causes you to be dismissed as someone who is unable to ever take anything or anyone else seriously. It's time to strengthen up your sense of what's really amusing and to improve your own sense of self worth.

\section{Steps}
\subsection{}
Start noticing when you laugh after people's comments, especially the clearly not funny comments. Try to be consciously aware of what the triggers are that cause you to laugh at these inopportune moments. Consider how you feel when it happens – are you feeling nervous, edgy, bored, argumentative, antipathetic to new ideas, uncomfortable or confused? Any negative feelings that bring about the need to laugh are an indicator that the laugh is serving to cover up or stand in for your true feelings. And those feelings will need to be tackled as well as the laughing habit (or compensation for nervousness).
\begin{itemize}
\item{Keep a journal of laughing triggers for a week. Can you see a pattern forming? Do you notice particular triggers?}
\end{itemize}

\subsection{}
Consider how others view your constant laugh, especially given that it comes at the tail-end of their usually ordinary comments. As noted in the introduction, there are different ways that your constant need to laugh after comments may be interpreted by people but eventually it's likely to wear thin and to cause some people to not take you very seriously. In a professional or working environment context, this is a serious outcome because it means that you'll risk being overlooked for promotions, plum roles and possibly even a pay raise. And when it comes to dating and love, nobody likes to hear someone laughing all the time they say something; it soon leads to them wondering if you're taking the relationship seriously and they may call things off. And all these interpretations occur just because you can't stop guffawing at casual comments! Tell yourself that you're definitely worth more than being seen as a lightweight by other people.
\begin{itemize}
\item{A good test is to think about is how the other people in the room will view you if you continue to laugh as they speak. If you are certain that it's fine to join in an obvious funny moment, then enjoy yourself. If they're likely to think that you are weird or out of sync with the situation, then try as hard as you can to control yourself.}
\end{itemize}

\subsection{}
This article isn't about repressing a good sense of humor and all laughter. There are clearly times when it remains both appropriate and tension-relieving to laugh. But it's important to know the difference between the habitual laugh and the real laugh born of genuine humor or happiness. You should know the difference just by feeling – a real laugh results because something is genuinely funny and the laughter leaves you feeling good and happy. The other kind of laugh is a compensatory mechanism aimed at covering up your unease. If you're not sure yet, ask yourself:
\begin{itemize}
\item{Does this laugh make me feel good or am I using it as a crutch to cover up negative feelings?}
\item{Is something about the situation making me feel uncomfortable or pressured?}
\item{Is my laughter infectious or are people look at me awkwardly and waiting for me to stop?}
\end{itemize}

\subsection{}
Become more assertive. Uncontrolled habitual laughing at the end of people's comments may be a sign that you feel inferior to other people and that you don't feel able to stand up for your own opinions and preferences. By laughing, there is a desire to seem inoffensive and even comforting to the speaker, as a way to mollify them so that they don't see you as an adversary. However, this will never bring about an enjoyable way of interacting with others and only serves to keep making you feel worse. Instead, learn to stand up for yourself and to be assertive, which is a non-confrontational, non-aggressive and non self-effacing way to interact with others. In learning to assert your needs with others in a fair, considerate and polite way, you will find that you see yourself as equal to every person and the inferiority sense will abate. And with it too, the trigger for the awkward laughter.

Think of witty or thoughtful replies that you could make to people instead of laughing. This might take some effort on your behalf but is well worth the trouble. Read up on the witty responses of people who say what they really want to say but in ways that are polite or clever. Learning some of their one-liners might help you to overcome the feeling of laughing and give you a real sense of finally voicing the things you'd rather say than simply cover up with giggling. If wit isn't your thing, try tactful truthfulness instead so that you can speak your mind without hurting the feelings of the other person.

\subsection{}
Find constructive ways to cope until you've overcome the habit of laughing at the end of everyone's comments. It takes time to change a habit, as well as determination but what do you do while you're aware that others are noticing your laughter? As you're working on stopping yourself from laughing after every comment, it's helpful to find some things to do that will cover up your laugh. For example, if you feel the urge coming over you, simply turn away and cover your mouth. Pretend it is a bit of hiccups, a sneeze or that there is a frog in your throat if anyone asks after you. Other ideas include:
\begin{itemize}
\item{Take deep breaths before allowing any sound to come out of your mouth. You can control laughing by deep breathing. This really works. Practice with friends.}
\item{Simply smile or nod your head.}
\item{Dig your fingernails into the palm of your hand to remind yourself that laughing is a habit that needs to be overcome, not indulged in.}
\item{Think of something serious if that works for you. Think of the unfinished in-tray, the dog poop you've yet to scoop off the back lawn or the time your boss threw your work back in your face. Those sorts of thoughts should put a dampener on the giggles.}
\end{itemize}

\subsection{}
Provide genuine outlets to release your laughter. It's important that you don't swing to extremes and become dull and overly serious! Find opportune, appropriate moments to thoroughly indulge your laughter. Spend time with happy, funny people cracking jokes, sharing funny stories and watching comedies together. Be prepared to see the lighter side of life all the time but learn to laugh on the inside when it's not appropriate to laugh out loud. Be a ray of sunshine in people's lives reminding them that laughter is a good tonic by being a source of funny things instead of hiding behind other people's comments – there is a world of difference between been a genuinely funny person who has a great sense of timing and a saddened soul with an inferiority complex who laughs out of neediness. Strive to use your real sense of humor to warm the hearts and minds of others and provoke their laughter. It's the best way to keep your laughter and to share it around!

\section{Tips}
\begin{itemize}
\item{Make sure you do laugh at funny things and not at serious things.}
\item{Be aware that laughter releases tension. Hence, people tend to laugh at really bizarre times, such as during attendance at a funeral. This is a way of relieving the tension and laughter and tears are so close that sometimes the signals get mixed up. Don't beat yourself up on such occasions; you're not alone and the best you can do is to quietly exit, have your tension-relieving laughing fit away from the occasion and return when you feel calmed and back to normal.}
\item{If you do laugh a lot, you aren't in trouble. It just means you need some adjustment as outlined above. But be careful not to put yourself down in the process of learning to undo your inappropriate laughing habit. For example, don't say, "OMG! I laugh too much. I must be so annoying!" The laugh doesn't exactly mean you are annoying but by drawing attention to it in this way, you label it for other people and they're likely to agree and you belittle yourself in the process. It's better not to say anything but if you feel you must, simply apologize, such as "Oh, I am sorry, that was inappropriate of me."}
\item{When you want to laugh just realize that nobody else is laughing, and focus on your breathing. Then after you do that for a while you should stop laughing at inappropriate times.
Laughing is good for your soul. Just make sure it's genuine laughter and not the nervous kind.}
\item{If you're worried you'll laugh during a meeting with someone, try to "laugh it all out" before meeting up. Think of something funny and laugh and laugh until you feel you can't anymore. Hopefully you'll feel all laughed out and won't be in the mood for laughing during your meeting.}
\end{itemize}

\section{Warnings}
\begin{itemize}
\item{Laughing at every little comment will cause people to see you as an airhead. Is that what you really want?}
\item{Don't exchange laughter for being boring and serious; it's important to laugh. Just be more in control of when you laugh instead of using it as a crutch}
\end{itemize}
\end{document}